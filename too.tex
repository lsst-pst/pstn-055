\subsection{ToO Time}\label{q:ToO}
The SCOC has looked favorably on a ToO program in its Phase 1 report and recommended it for simulation as a microsurvey (\autoref{q:Micro}). The questions left open after the Phase 1 recommendations were:
\begin{itemize}
\item How many Targets of Opportunity (ToOs) per year should be observed?
\item How should time be allocated for ToOs with respect to the LIGO-Virgo runs?
\item How should ToO observations be coordinated with other groups?
\item Should ToO observations fall only on the night of the trigger or should follow-up continue on later days?
\end{itemize} 



The envelope of time required to implement a ToO program with Rubin is well-defined thanks to community studies \citep{https://doi.org/10.48550/arxiv.1812.04051, Andreoni_2022}. These studies indicate that 2--3\% of the LSST time dedicated to ToOs would enable effective imaging follow-up of Gravitational Wave triggers, given their expected event rates and sky localization area. The overall impact on the WFD is likely to be less than the above allocation since the majority of the ToO observations would fall within the WFD area and may be used within the WFD (depending on the details of the ToO observing strategy such as exposure time and dither). Following the analysis of simulations, the SCOC believes a ToO program for the follow-up of Gravitational Waves is a good investment of 2-3\% of LSST time, but the implementation details for this program remain to be defined and the answers to the questions above depend on the program implementation.

To enable a ToO program with Rubin, strategic decisions have to be made ahead of time so that the observations can be deployed within the scheduler with minimal-to-no human intervention. The following questions, thus, need to be answered:
\begin{enumerate}
\item How will targets for follow-up be chosen?
\item What observing strategy should be implemented for each target (\emph{e.g.}, exposure, filters, cadence, duration), possibly depending on trigger characteristics?

\end{enumerate}

Answering these questions requires close collaboration between the Multi-Messenger astronomy community and members of Rubin Observatory in order to identify what is scientifically ideal (as the existing community studies did), what is technically possible, and where the two meet. 

{\bf With the existing community studies as guidance, the SCOC recommends that a ToO program to respond to Gravitational Waves and potentially other Multi-Messenger astronomy triggers be established. The SCOC recommends that this program be contained to $\leq$3\% of the LSST time. The SCOC recommends that Rubin organizes a workshop in 2023 to bring together members of the scientific community, members of Rubin Observatory (including observing and scheduler specialists, and Data Management specialists) and members of the SCOC to define the details of the implementation of the Rubin ToO program. This workshop should produce a document detailing recommendations for implementation, including suggestions for the questions outlined above, that the experts agree would accomplish the scientific goals of the program.}


The workshop shall be well advertised and open to all members of the broader scientific community that desire to contribute to this study. A template of the document shall be released to the workshop participants ahead of time, detailing the guidance necessary to implement the ToO program in years 1-2 of LSST (we note that the year 1 plan may differ significantly from other years as do other aspects of the survey, see \autoref{q:Early} and \citeds{rtn-011}). 

Further SCOC recommendations (for example on how to distribute ToO time with respect to the LIGO-Virgo runs) should be delayed till after receipt of the aforementioned document. 

The scientific case for Rubin ToOs has been best demonstrated for Multi-Messenger events and should serve as a case study to develop a Rubin ToO program. The SCOC acknowledges the possibility that other scientifically pressing cases for ToOs with Rubin may appear in the future. Once the framework for enabling ToOs exists and has been designed and tuned to the currently compelling Multi-Messenger follow-up scientific case, {\bf the SCOC recommends that a process for requesting and approving non-Multi-Messenger ToO observations of rare and compelling phenomena that require the unique capabilities of Rubin be established}.
