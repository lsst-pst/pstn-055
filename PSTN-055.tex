\documentclass[PST,authoryear,toc]{lsstdoc}
\input{meta}

% Package imports go here.
\usepackage{graphicx} %Loading the package
\graphicspath{{figures/}}
% Local commands go here.
\usepackage[export]{adjustbox}

\usepackage{placeins}

% Local commands go here.

\newcommand{\question}[1]{{\color{red} #1}}
\newcommand{\roughdraft}[1]{{\color{blue} #1}}
\newcommand{\predraftnotes}[1]{{\color{orange} #1}}
\newcommand{\degsq}{\ensuremath{\mathrm{deg}^2}}
\newcommand{\ebv}{\ensuremath{E(\mathrm{B-V})}}
\newcommand{\opsim}{\texttt{OpSim}}
\newcommand{\degree}{\ensuremath{^{\circ}}}
\defcitealias{PSTN-053}{PSTN-053}
\defcitealias{PSTN-051}{PSTN-051}
\defcitealias{LPM-17}{SRD}

%If you want glossaries
%\input{aglossary.tex}
%\makeglossaries

\title{Survey Cadence Optimization Committee’s Phase 2 Recommendations}

% Optional subtitle
% \setDocSubtitle{A subtitle}

\author{%
{The Rubin Observatory Survey Cadence Optimization Committee}}

\setDocRef{PSTN-055}
\setDocUpstreamLocation{\url{https://github.com/lsst-pst/pstn-055}}

\date{\vcsDate}

% Optional: name of the document's curator
% \setDocCurator{Federica B. Bianco}

\setDocAbstract{%
Survey Cadence Optimization Committee’s Phase 2 Recommendations
}

% Change history defined here.
% Order: oldest first.
% Fields: VERSION, DATE, DESCRIPTION, OWNER NAME.
% See LPM-51 for version number policy.
\setDocChangeRecord{%
  \addtohist{1}{2022-12-31}{Released.}{Federica Bianco}
}

\setDocAbstract{

This document describes the recommendations formulated by the Survey Cadence Optimization Committee (SCOC) of Vera C. Rubin Observatory for the observing strategy of the Legacy Survey of Space and Time (LSST). These ``Phase 2'' recommendations follow the initial ``Phase 1'' recommendations released in December 2021  \citep[][2022]{PSTN-053} and convey the current recommendations for the initial implementation of the LSST. These recommendations are based on the SCOC analysis of LSST simulations generated according to \citetalias{PSTN-053} and the feedback received from the scientific community on those simulations. These recommendations should be considered as the current guidance for implementing the LSST observing survey at the start of the survey, a strategy to be reviewed regularly over the survey lifetime, and are accompanied by a simulation, \texttt{baseline\_v3.0}, which represents a specific implementation of these recommendations. While these recommendations dramatically narrow the possible LSST observing parameter choices, we emphasize that at this stage they do not yet represent a final or complete set of recommendations for the survey strategy, and several details are still to be refined. Furthermore, the strategy should be considered ``living'' and subject to evolution throughout the survey. The SCOC continues to engage with the community to finalize its recommendations. At the time of the release of this document, the final phases of construction of Rubin Observatory are ongoing with an expected first light in mid-2024. We expect the SCOC recommendations to be revised as needed, as knowledge of the system as built and the outcome of commissioning and Science Verification mature. This document describes the Phase 2 SCOC recommendations ---highlighting elements that can be considered final and elements that remain to be finalized---, the process through which the SCOC has converged on these recommendations, and the working plan for the SCOC starting in January 2023 through the start of the survey to finalize the initial implementation of the Legacy Survey of Space and Time.


}

\begin{document}

% Create the title page.
\maketitle
% Frequently for a technote we do not want a title page  uncomment this to remove the title page and changelog.
% use \mkshorttitle to remove the extra pages

% ADD CONTENT HERE
% You can also use the \input command to include several content files.

\section{Introduction}

This document describes a set of recommendations for the initial implementation of the Vera C. Rubin Observatory's Legacy Survey of Space and Time (LSST) observing strategy. These recommendations were produced by Rubin Observatory's Survey Cadence Optimization Committee (SCOC) as ``Phase 2'' recommendations and are intended as guidance to Rubin Observatory's Director to implement the LSST observations. It should be noted, and it is discussed in detail in this document, that aspects of the observing strategy are still under refinement, and that the observing strategy should be continuously reevaluated during operation to ensure it continues to maximize the scientific throughput of the survey.

The present document assumes the reader has a basic familiarity with Rubin Observatory, the LSST, and the process of converging to an observing strategy that will achieve the many scientific goals of the survey. The reader is referred to documents including \citealt{LPM-17}, \citealt{PSTN-051}, \citealt{PSTN-053}, and \citet{2022ApJS..258....1B}, among others, to acquire the relevant background information. 

The SCOC is a standing committee instituted in 2018 as an advisory body to Rubin Observatory's Director of Operations with the charge to:
\begin{itemize}
\item Recommend an initial survey strategy for the Rubin Legacy Survey of Space and Time, which includes defining:
\begin{itemize}
\item Footprint to be observed in the Wide Fast Deep Survey (hereafter, WFD),
\item Footprint to be observed in ``special regions'' outside of the WFD,
\item Cadence for observing the WFD,
\item Cadence for observing special regions and time to be spent on special regions,
\item Filter balance over the WFD,
\item Filter balance over the rest of the footprint,
\item Selection of the Deep Drilling Fields (hereafter DDFs),
\item Time to be spent on the DDFs,
\item Cadence for observing DDFs,
\item Propose evaluation mechanisms for adoption and prioritization of community proposals for surveys beyond the WFD.
\end{itemize}
\item Develop an ``Early Science'' plan.
\item Review the survey strategy during Operations on a regular basis and recommend adjustments.
\end{itemize}

The detailed charge of the SCOC and its membership can be found at \url{https://www.lsst.org/content/charge-survey-cadence-optimization-committee-scoc}. In reference to the development of an early science plan, we note that the first year of the LSST is likely to require a different strategy implementation than the rest of the survey and detail of the observing strategy in year 1 remain be defined by the SCOC and Rubin Operations team (see \citealt{rtn-011} and \autoref{q:Early}).

Over the past decade, Rubin Observatory has engaged the community in the design of the details of Rubin LSST, beyond the broad constraints imposed by the four core science deliverables of Rubin as described in the Science Requirements Document by \citet[][hereafter SRD]{LPM-17}, in a process which is described in \citet{2022ApJS..258....1B}. 

After its formation, the SCOC solicited %\footnote{Through a call available at \url{https://docs.google.com/document/d/1SYKsJkNVIGE1t5vbZ_L0qLLsV7UHeraXNVZ22-xFk4c}.}
and reviewed 39 Cadence Notes\footnote{\label{fn:cnotes}Available at \url{https://www.lsst.org/content/survey-cadence-notes-2021}.} proposing changes and enhancements to the then current plan for LSST, and a large number of LSST simulations (created through the \opsim\ framework as described in \citealt[][hereafter PSTN-051]{PSTN-051}). Collecting community feedback and metrics, the SCOC ``Phase 1'' process culminated in the recommendations released in \citetalias[or MAF][]{PSTN-053} and hundreds of v2.X simulations produced by the Survey Strategy team of Rubin Observatory that reflect these recommendations and vary the parameters yet to be finalized. 
Through December 2021 and beyond, the Survey Strategy Team, as charged by the SCOC, has also supported the community through the process of producing and integrating science-driven evaluations of the simulations into the \texttt{rubin\_sim} software\footnote{\url{https://github.com/lsst/rubin_sim}.}. These simulations and science-driven metrics produced within the Metric Analysis Framework \citep{2014SPIE.9149E..0BJ} or \texttt{MAF},  constitute a core input utilized by the SCOC, together with feedback received from the community in a variety of ways, to produce the Phase 2 recommendations described below. 




\subsection{Brief Synopsis of Phase 2 recommendations}\label{sec:shortrec}

While additional work is needed to finalize a plan for the initial LSST survey strategy (see \autoref{sec:refinements} for a summary of forthcoming refinements), the set of recommendations we present here narrows the parameter space of the possible LSST strategies substantially. Based on evaluations of possible observing strategies demonstrated in simulations from \opsim\ v2.0 through v2.99 (see \autoref{ssec:process}), the SCOC is recommending a survey strategy with the following characteristics:



\setlength\parindent{0.7cm}
\hangindent=0.7cm A WFD low-dust region definition with limits $−70\degree \leq$ Dec $< +15\degree$ for
RA $\sim 7–18$~h and $−70\degree \leq$ Dec $<+3\degree$ for 0 $\lesssim$ RA $\lesssim$ 7~h and 18~h $\lesssim$ RA $\lesssim$  24~h,  and the addition of the Virgo cluster in the WFD survey. The current recommendation for the Galactic Plane and Bulge coverage is driven by the expert advice of the SMWLV and TVS Science Collaborations but enforces higher contiguity in space coverage to better exploit specific Rubin strengths and capabilities, and improve survey efficiency. We encourage the Galactic science community to continue to work with the SCOC to finalize the survey footprint on the Galactic sky. The fraction of time spent in each of the low-dust WFD footprint, the Galactic sky, the South Celestial Pole (SCP), and the North Ecliptic Spur (NES) remains close to the fractions recommended in Phase 1 (see \citealt{PSTN-053}, section 2), and for each of these surveys the \texttt{baseline\_v3.0} implementation is within 2\% of \texttt{baseline\_v2.0} (\autoref{q:Footprint}). 

\hangindent=0.7cm Maintaining the filter balance as was implemented in \texttt{baseline\_v2.0} for the low-dust WFD region and, tentatively, for the Galactic Plane/Bulge and other special regions. However, we  encourage members of the scientific community with specific interest and expertise in the science performed with data from the special surveys to evaluate if a modified filter balance in these regions could better support the desired science outcomes (\autoref{q:Filters}).

\hangindent=0.7cm Visit pairs are maintained with a $\sim33$ minute time gap, to be collected in different filters (the filter matches may be subject to further refinement). The SCOC recommends that the LSST cadence be designed to ensure coverage of time scales in the hours-to-one-day range generally lacking in most simulations prior to v2.99.
This can be achieved with a third nightly visit at several hours separation once every several ($\sim7$) nights and increasing the probability of visits in the following night (in one of the bands that were already observed). This strategy, when implemented jointly with rolling, provides coverage at time scales from a few to 30 hours (\autoref{q:Visits}) which would otherwise be undersampled.  

\hangindent=0.7cm Rolling, \emph{i.e.} a cadence where a portion of the sky is emphasized at a point in time, to then be de-emphasized later, is recommended on the WFD, with the sky split into two 1/2-sky regions defined by declination limits and with a 0.9 rolling weight. This recommendation is made under the assumption that sufficient uniformity in depth to support static-sky cosmology can be achieved with a software solution.  Should this not be the case, the SCOC will re-evaluate this recommendation (\autoref{q:Rolling}).

\hangindent=0.7cm That no less than 5\% of the survey, with the potential to increase this to as high as 7\%, be dedicated to observing 5 pointings as LSST Deep Drilling Fields (DDFs)\footnote{See \url{https://www.lsst.org/scientists/survey-design/ddf}.}. In the current simulations, with this time investment the DDFs will generally reach a 10-year coadded depths $\sim1.3$ magnitudes deeper in each band than the coadded depth of a WFD pointing in the same band. To select the Euclid Deep Field South as the fifth LSST DDF with a footprint that can be covered with two LSST pointings, each to be observed at 0.5 of the 10-year DDF depth;  that all DDFs be observed for 10 years and the COSMOS field be observed to full 10-year depth within the first 3 years of LSST and continue to be observed thereafter at the same rate as the other DDFs. The detailed intra-night strategy on the DDFs is still under design (\autoref{q:DDF}).

\hangindent=0.7cm The SCOC recommends that two microsurveys are scheduled  in year 1: the near-sun NEO twilight survey and, if time is
available, the Northern Strip survey. Additional microsurveys should be added in the future, when the system characteristics and survey efficiency are better assessed, and a process is recommended to receive and review refined and additional microsurvey proposals after the beginning of LSST (\autoref{q:Micro}).

\hangindent=0.7cm The SCOC recommends a Target of Opportunity (ToO) program be enabled to respond to Gravitational Waves and Multi-Messenger Astronomy triggers with a fraction of $\leq 3\%$ of dedicated survey time, with the possibility of extending it to additional types of targets in the future; a path toward a process to formulate recommendations for target selection and observing strategy details is outlined (\autoref{q:ToO}).

\hangindent=0.7cm Finally, the SCOC commits to working with the Operations team in 2023 to establish the best strategy for Early Science (see \citealt{rtn-011}). Optimizing the LSST year 1 observing schedule for early science may mean that the time  sampling will look somewhat different from that in subsequent years (\autoref{q:Early}).

\noindent The SCOC will continue to refine the recommendations presented here and work closely with Rubin Observatory's Project and Operation teams to update the recommendation with knowledge of the system as built and the outcomes of Commissioning and Science Verification.



\appendix
% Include all the relevant bib files.
% https://lsst-texmf.lsst.io/lsstdoc.html#bibliographies
% Include all the relevant bib files.
% https://lsst-texmf.lsst.io/lsstdoc.html#bibliographies
\section{Acknowledgements}
The SCOC acknowledges significant efforts and expresses immense gratitute to many dedicated members of the Rubin LSST Science Collaborations and broader scientific community who contributed to the optimization of the Rubin LSST Survey Strategy for many years and continue to actively engage with the SCOC on this topic. 

The SCOC is grateful to the Survey Strategy team who supported the SCOC work and the community contribution to the optimization of the Rubin LSST survey strategy: 
Dr. Peter Yoachim, 
Dr. Eric Neilson,
Dr. Tiago Ribeiro, and Dr. Lynne Jones.

Rubin Observatory Construction Project and Operations acknowledge and thank the SCOC members for their significant and sustained   work toward the design of the LSST observing strategy and in particular express their gratitude to the outgoing members of the committee: Dr. Sarah Brough, Dr. Renée Hložek, Dr. Mansi Kasliwal, Dr. Hiranya Peiris, Dr. Meg Schwamb, Dr. Dan Scolnic.

\subsection{SCOC membership}

During the Phase 2 deliberations, the SCOC composition was as follows:

 Franz Bauer, Universidad Católica, Chile

 Federica Bianco, Rubin Observatory/University of Delaware (ex officio, chair)

 Sarah Brough, University of New South Wales 

 Renee Hlozek, University of Toronto


 Lynne Jones, Rubin Observatory (ex officio)

 Mansi Kasliwal, Caltech 

 Knut Olsen, NSF’s NOIRLab 

 Hiranya Peiris, University College London/Stockholm University

 Meg Schwamb, Queen's University Belfast

 Dan Scolnic, Duke University

 Colin Slater, University of Washington  

 Jay Strader, Michigan State University


\section{References} \label{sec:bib}
\renewcommand{\refname}{} % Suppress default Bibliography section
\bibliography{local,lsst,lsst-dm,refs_ads,refs,books}

% Make sure lsst-texmf/bin/generateAcronyms.py is in your path
\section{Acronyms} \label{sec:acronyms}
\addtocounter{table}{-1}
\begin{longtable}{p{0.145\textwidth}p{0.8\textwidth}}\hline
\textbf{Acronym} & \textbf{Description}  \\\hline

DM & Data Management \\\hline
\end{longtable}

% If you want glossary uncomment below -- comment out the two lines above
%\printglossaries





\end{document}
