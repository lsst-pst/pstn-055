\documentclass[PST,authoryear,toc]{lsstdoc}
\input{meta}

% Package imports go here.
\usepackage{graphicx} %Loading the package
\graphicspath{{figures/}}
% Local commands go here.
\usepackage[export]{adjustbox}

\usepackage{placeins}

% Local commands go here.

\newcommand{\question}[1]{{\color{red} #1}} 
\newcommand{\roughdraft}[1]{{\color{blue} #1}}
\newcommand{\predraftnotes}[1]{{\color{orange} #1}}
\newcommand{\degsq}{\ensuremath{\mathrm{deg}^2}}
\newcommand{\ebv}{\ensuremath{E(\mathrm{B-V})}}
\newcommand{\opsim}{\texttt{OpSim}}
\newcommand{\degree}{\ensuremath{^{\circ}}}
\defcitealias{PSTN-053}{PSTN-053}
\defcitealias{PSTN-051}{PSTN-051}
\defcitealias{LPM-17}{SRD}

%If you want glossaries
%\input{aglossary.tex}
%\makeglossaries

\title{Survey Cadence Optimization Committee’s Phase 2 Recommendations}

% Optional subtitle
% \setDocSubtitle{A subtitle}

\author{%
The Rubin Observatory Survey Cadence Optimization Committee}

\setDocRef{PSTN-055}
\setDocUpstreamLocation{\url{https://github.com/lsst-pst/pstn-055}}

\date{\vcsDate}

% Optional: name of the document's curator
% \setDocCurator{Federica B. Bianco}

\setDocAbstract{%
Survey Cadence Optimization Committee’s Phase 2 Recommendations
}

% Change history defined here.
% Order: oldest first.
% Fields: VERSION, DATE, DESCRIPTION, OWNER NAME.
% See LPM-51 for version number policy.
\setDocChangeRecord{%
  \addtohist{1}{2022-12-31}{Released.}{Federica Bianco}
}

\setDocAbstract{

This document describes the recommendations formulated by the Survey Cadence Optimization Committee (SCOC) of Vera C. Rubin Observatory for the observing strategy of the Legacy Survey of Space and Time (LSST). These ``Phase 2'' recommendations follow the initial ``Phase 1'' recommendations released in December 2021 as \citeds{PSTN-053} and convey the current recommendations for the initial implementation of the LSST. These recommendations are based on the SCOC analysis of LSST simulations generated according to \citetalias{PSTN-053} and the feedback received from the scientific community on those simulations. These recommendations should be considered as the current guidance for implementing the LSST observing survey at the start of the survey, a strategy to be reviewed regularly over the survey lifetime, and are accompanied by a simulation, \texttt{baseline\_v3.0}, which represents a specific implementation of these recommendations. While these recommendations dramatically narrow the possible LSST observing parameter choices, we emphasize that at this stage they do not yet represent a final or complete set of recommendations for the survey strategy, and several details are still to be refined. Furthermore, the strategy should be considered ``living'' and subject to evolution throughout the survey. The SCOC continues to engage with the community to finalize its recommendations. At the time of the release of this document, the final phases of construction of Rubin Observatory are ongoing with an expected first light in mid-2024. We expect the SCOC recommendations to be revised as needed, as knowledge of the system as built and the outcome of commissioning and Science Verification mature. This document describes the Phase 2 SCOC recommendations ---highlighting elements that can be considered final and elements that remain to be finalized---, the process through which the SCOC has converged on these recommendations, and the working plan for the SCOC starting in January 2023 through the start of the survey to finalize the initial implementation of the Legacy Survey of Space and Time.


}

\begin{document}

% Create the title page.
\maketitle
% Frequently for a technote we do not want a title page  uncomment this to remove the title page and changelog.
% use \mkshorttitle to remove the extra pages

% ADD CONTENT HERE
% You can also use the \input command to include several content files.

\section{Introduction}

This document describes a set of recommendations for the initial implementation of the Vera C. Rubin Observatory's Legacy Survey of Space and Time (LSST) observing strategy. These recommendations were produced by Rubin Observatory's Survey Cadence Optimization Committee (SCOC) as ``Phase 2'' recommendations and are intended as guidance to Rubin Observatory's Director to implement the LSST observations. It should be noted, and it is discussed in detail in this document, that aspects of the observing strategy are still under refinement, and that the observing strategy should be continuously reevaluated during operation to ensure it continues to maximize the scientific throughput of the survey.

The present document assumes the reader has a basic familiarity with Rubin Observatory, the LSST, and the process of converging to an observing strategy that will achieve the many scientific goals of the survey. The reader is referred to documents including \citealt{LPM-17}, \citealt{PSTN-051}, \citealt{PSTN-053}, and \citet{2022ApJS..258....1B}, among others, to acquire the relevant background information. 

The SCOC is a standing committee instituted in 2018 as an advisory body to Rubin Observatory's Director of Operations with the charge to:
\begin{itemize}
\item Recommend an initial survey strategy for the Rubin Legacy Survey of Space and Time, which includes defining:
\begin{itemize}
\item Footprint to be observed in the Wide Fast Deep Survey (hereafter, WFD),
\item Footprint to be observed in ``special regions'' outside of the WFD,
\item Cadence for observing the WFD,
\item Cadence for observing special regions and time to be spent on special regions,
\item Filter balance over the WFD,
\item Filter balance over the rest of the footprint,
\item Selection of the Deep Drilling Fields (hereafter DDFs),
\item Time to be spent on the DDFs,
\item Cadence for observing DDFs,
\item Propose evaluation mechanisms for adoption and prioritization of community proposals for surveys beyond the WFD.
\end{itemize}
\item Develop an ``Early Science'' plan.
\item Review the survey strategy during Operations on a regular basis and recommend adjustments.
\end{itemize}

The detailed charge of the SCOC and its membership can be found at \url{https://www.lsst.org/content/charge-survey-cadence-optimization-committee-scoc}. In reference to the development of an early science plan, we note that the first year of the LSST is likely to require a different strategy implementation than the rest of the survey and detail of the observing strategy in year 1 remain be defined by the SCOC and Rubin Operations team (see \citealt{rtn-011} and \autoref{q:Early}).

Over the past decade, Rubin Observatory has engaged the community in the design of the details of Rubin LSST, beyond the broad constraints imposed by the four core science deliverables of Rubin as described in the Science Requirements Document by \citet[][hereafter SRD]{LPM-17}, in a process which is described in \citet{2022ApJS..258....1B}. 

After its formation, the SCOC solicited %\footnote{Through a call available at \url{https://docs.google.com/document/d/1SYKsJkNVIGE1t5vbZ_L0qLLsV7UHeraXNVZ22-xFk4c}.}
and reviewed 39 Cadence Notes\footnote{\label{fn:cnotes}Available at \url{https://www.lsst.org/content/survey-cadence-notes-2021}.} proposing changes and enhancements to the then current plan for LSST, and a large number of LSST simulations (created through the \opsim\ framework as described in \citealt[][hereafter PSTN-051]{PSTN-051}). Collecting community feedback and metrics, the SCOC ``Phase 1'' process culminated in the recommendations released in \citetalias[or MAF][]{PSTN-053} and hundreds of v2.X simulations produced by the Survey Strategy team of Rubin Observatory that reflect these recommendations and vary the parameters yet to be finalized. 
Through December 2021 and beyond, the Survey Strategy Team, as charged by the SCOC, has also supported the community through the process of producing and integrating science-driven evaluations of the simulations into the \texttt{rubin\_sim} software\footnote{\url{https://github.com/lsst/rubin_sim}.}. These simulations and science-driven metrics produced within the Metric Analysis Framework \citep{2014SPIE.9149E..0BJ} or \texttt{MAF},  constitute a core input utilized by the SCOC, together with feedback received from the community in a variety of ways, to produce the Phase 2 recommendations described below. 




\subsection{Brief Synopsis of Phase 2 recommendations}\label{sec:shortrec}

While additional work is needed to finalize a plan for the initial LSST survey strategy (see \autoref{sec:refinements} for a summary of forthcoming refinements), the set of recommendations we present here narrows the parameter space of the possible LSST strategies substantially. Based on evaluations of possible observing strategies demonstrated in simulations from \opsim\ v2.0 through v2.99 (see \autoref{ssec:process}), the SCOC is recommending a survey strategy with the following characteristics:



\setlength\parindent{0.7cm}
\hangindent=0.7cm A WFD low-dust region definition with limits $−70\degree \leq$ Dec $< +15\degree$ for
RA $\sim 7–18$~h and $−70\degree \leq$ Dec $<+3\degree$ for 0 $\lesssim$ RA $\lesssim$ 7~h and 18~h $\lesssim$ RA $\lesssim$  24~h,  and the addition of the Virgo cluster in the WFD survey. The current recommendation for the Galactic Plane and Bulge coverage is driven by the expert advice of the SMWLV and TVS Science Collaborations but enforces higher contiguity in space coverage to better exploit specific Rubin strengths and capabilities, and improve survey efficiency. We encourage the Galactic science community to continue to work with the SCOC to finalize the survey footprint on the Galactic sky. The fraction of time spent in each of the low-dust WFD footprint, the Galactic sky, the South Celestial Pole (SCP), and the North Ecliptic Spur (NES) remains close to the fractions recommended in Phase 1 (see \citealt{PSTN-053}, section 2), and for each of these surveys the \texttt{baseline\_v3.0} implementation is within 2\% of \texttt{baseline\_v2.0} (\autoref{q:Footprint}). 

\hangindent=0.7cm Maintaining the filter balance as was implemented in \texttt{baseline\_v2.0} for the low-dust WFD region and, tentatively, for the Galactic Plane/Bulge and other special regions. However, we  encourage members of the scientific community with specific interest and expertise in the science performed with data from the special surveys to evaluate if a modified filter balance in these regions could better support the desired science outcomes (\autoref{q:Filters}).

\hangindent=0.7cm Visit pairs are maintained with a $\sim33$ minute time gap, to be collected in different filters (the filter matches may be subject to further refinement). The SCOC recommends that the LSST cadence be designed to ensure coverage of time scales in the hours-to-one-day range generally lacking in most simulations prior to v2.99.
This can be achieved with a third nightly visit at several hours separation once every several ($\sim7$) nights and increasing the probability of visits in the following night (in one of the bands that were already observed). This strategy, when implemented jointly with rolling, provides coverage at time scales from a few to 30 hours (\autoref{q:Visits}) which would otherwise be undersampled.  

\hangindent=0.7cm Rolling, \emph{i.e.} a cadence where a portion of the sky is emphasized at a point in time, to then be de-emphasized later, is recommended on the WFD, with the sky split into two 1/2-sky regions defined by declination limits and with a 0.9 rolling weight. This recommendation is made under the assumption that sufficient uniformity in depth to support static-sky cosmology can be achieved with a software solution.  Should this not be the case, the SCOC will re-evaluate this recommendation (\autoref{q:Rolling}).

\hangindent=0.7cm That no less than 5\% of the survey, with the potential to increase this to as high as 7\%, be dedicated to observing 5 pointings as LSST Deep Drilling Fields (DDFs)\footnote{See \url{https://www.lsst.org/scientists/survey-design/ddf}.}. In the current simulations, with this time investment the DDFs will generally reach a 10-year coadded depths $\sim1.3$ magnitudes deeper in each band than the coadded depth of a WFD pointing in the same band. To select the Euclid Deep Field South as the fifth LSST DDF with a footprint that can be covered with two LSST pointings, each to be observed at 0.5 of the 10-year DDF depth;  that all DDFs be observed for 10 years and the COSMOS field be observed to full 10-year depth within the first 3 years of LSST and continue to be observed thereafter at the same rate as the other DDFs. The detailed intra-night strategy on the DDFs is still under design (\autoref{q:DDF}).

\hangindent=0.7cm The SCOC recommends that two microsurveys are scheduled  in year 1: the near-sun NEO twilight survey and, if time is
available, the Northern Strip survey. Additional microsurveys should be added in the future, when the system characteristics and survey efficiency are better assessed, and a process is recommended to receive and review refined and additional microsurvey proposals after the beginning of LSST (\autoref{q:Micro}).

\hangindent=0.7cm The SCOC recommends a Target of Opportunity (ToO) program be enabled to respond to Gravitational Waves and Multi-Messenger Astronomy triggers with a fraction of $\leq 3\%$ of dedicated survey time, with the possibility of extending it to additional types of targets in the future; a path toward a process to formulate recommendations for target selection and observing strategy details is outlined (\autoref{q:ToO}).

\hangindent=0.7cm Finally, the SCOC commits to working with the Operations team in 2023 to establish the best strategy for Early Science (see \citealt{rtn-011}). Optimizing the LSST year 1 observing schedule for early science may mean that the time  sampling will look somewhat different from that in subsequent years (\autoref{q:Early}).

\noindent The SCOC will continue to refine the recommendations presented here and work closely with Rubin Observatory's Project and Operation teams to update the recommendation with knowledge of the system as built and the outcomes of Commissioning and Science Verification.


\subsection{The process of formulating Phase 2 recommendations}\label{ssec:process}

%The current recommendations presented here constitute ``Phase 2'' of the LSST strategy optimization process and substantially narrow the parameter space of the possible LSST strategy options. 

To converge to the recommendations presented herein, the SCOC has reviewed close to 500 simulations (v2.X) and hundreds of metrics, including metrics evaluating the impact of the cadence choices on the system, compliance with the \citepalias{LPM-17}, and performance on community-contributed science goals. The simulations were released in batches starting in November 2021, and they are described in detail in a  community.lsst.org post\footnote{\url{https://community.lsst.org/} is an online forum used by Rubin for discussions and announcements. The specific post describing the v2.X simulations is \url{https://community.lsst.org/t/survey-simulations-v2-1-april-2022/6538}} and on GitHub.\footnote{\url{https://github.com/lsst-pst/survey_strategy/blob/main/fbs_2.0/SummaryInfo_v2.1.ipynb}.}

After reviewing \opsim\ v2.X, the SCOC commissioned the production of and reviewed $\sim 10$ simulations (\opsim\ version v2.99) that largely straddled the remaining survey strategy options. An initial set of four simulations were released on October 24, 2022\footnote{\url{ https://community.lsst.org/t/draft-of-v3-0-survey-strategies-v2-99/7159}.} and were presented at the Third SCOC-Science Collaborations Workshop on November 2-3, 2022.\footnote{\url{https://project.lsst.org/meetings/scoc-sc-workshop3/home}.}  Additional simulations refined the v2.99 initial set of four and implemented suggestions collected during and after the November 2022 workshop. Among those simulations, we selected the one that best represents our current recommendation as \texttt{baseline\_v3.0} (described in \autoref{sec:v3})

The detailed process of reviewing input and converging to an optimal recommendation is complex, because, in this context, ``optimal'' is difficult to define. While this process of generating survey simulations and associated metrics does provide quantitative measures of enhancement of the survey, it should be noted that the metrics are not always directly comparable, nor is the overall importance of the science they reflect an objectively measurable quantity, such that a purely numerical optimization process is not possible. The performance of the strategy on different science drivers has to be balanced in the light of (1) the output of available metrics, but also (2) expert considerations about the relative importance of the performance gain/loss for different science drivers, (3) the significance of the performance change (\emph{i.e.} the metrics are not ``standardized'' in a collective sense as they measure quantities with different units and that may not be directly comparable), and (4) how core a science goal is to the overall Rubin LSST science endeavor. At a high level, the SCOC reviews the increase/decrease in performance for a science case as compared to the earlier LSST plans, known as the ``baseline'' (in the case of the Phase 2 recommendations, comparing the \texttt{v2.0} and \texttt{v2.1} simulations with the \texttt{baseline\_v1.7}, and  the simulations that implement the current recommendations ---\texttt{v2.99} and \texttt{v3.0}--- with \texttt{baseline\_v2.X}). Generally, performance changes on metrics greater than a few percent are regarded as significant. Further refinement is beyond the current scope of the SCOC work, particularly considering the as-of-yet unknown performance of the system as built. Finally, the process of optimizing the survey is further complicated by the fact that the SCOC is provided with LSST families of simulations that aim to stretch the survey in one or another direction (\emph{e.g.}, modifying the footprint or the rolling scheme). This is typically done in isolation, but the SCOC needs to keep track of and balance the compound effects of survey changes across multiple parameters, which might significantly impact a science case even when individually they had a small effect. 

The SCOC does not only review the MAF metrics performance but also provides \emph{key interpretability} to the metrics and their performance. For this the SCOC is composed of experts in different domains, attempting to cover all science areas relevant to Rubin. It should be noted, however, that SCOC members are explicitly instructed not to view themselves as advocates of specific science areas, but rather as members exercising their best judgment on the LSST observing strategy with the goal of maximizing the \emph{overall} scientific throughput of the survey.

The SCOC has also liaised with the eight Science Collaborations (SCs) of LSST, with 1--3 members of the SCOC assigned to each SC as liaisons.\footnote{See \url{https://www.lsst.org/content/charge-survey-cadence-optimization-committee-scoc} for the full list of SCOC liaisons.} In this role, SCOC members are charged with enabling and ensuring a bidirectional communication flow between the SCOC and the SCs. 

The SCOC is committed to transparency. Communication between the SCOC and the community is made through a number of channels: via liaisons to the SCs; summaries of the SCOC goals and discussions on community.lsst.org;\footnote{See the dedicated ``Survey Strategy'' topic (\url{https://community.lsst.org/c/sci/survey-strategy/37}).} presentations of our current work and status of the recommendations at relevant meetings, including the annual Project Community Workshop (PCW); and by organizing dedicated workshops to come together with the community. Most recently the SCOC and community met in the Third SCOC Workshop on November 2-3, 2022.\footnote{\url{https://project.lsst.org/meetings/scoc-sc-workshop3/home}.} Phase 2 recommendation updates have been released in a post on community.lsst.org\footnote{\url{https://community.lsst.org/t/scoc-v2-0-and-2-1-simulations-review-timeline/6712}.} leading to the November workshop, including short summaries of the discussion as conducted in each full SCOC meeting.\footnote{Released in \url{https://community.lsst.org/t/scoc-v2-0-and-2-1-simulations-review-timeline/6712}.} Starting in November 2022, a dedicated community post is regularly updated with meeting summaries and future meetings' schedules,\footnote{\url{https://community.lsst.org/t/public-scoc-meeting-minutes/7185}.} and regular office hours have been established.\footnote{\url{https://community.lsst.org/t/scoc-office-hour/7221}.}

While we had solicited official reports in the SCOC Phase 1 deliberations (the 2019 Cadence notes,\footnote{See \autoref{fn:cnotes}.}
the SCOC decided shortly after the v2.0 simulations were made available {\it not} to solicit formal reports from the SCs and the community in the Phase 2 deliberations so as to avoid overburdening the community members. Yet several SCs shared their feedback in a number of written reports. Reports received by the SCOC in its Phase 2 deliberations include analyses of the \opsim\ via metrics relevant to a specific science-focused group, as well as  considerations not based on metrics but on expert opinions from domain scientists. All these reports are made available to the reader.\footnote{Listed in the order in which they were received, these include reports from the
SSSC (May 2022);
the TVS and SMWLV SCs (a report on Galactic science and a more specific report on microlensing), TVS SC (on exctra-galactic science), and TVS and AGN SCs (the latter four all first received in March 2022); three 
DESC reports (June 2022, October 2022, and November 2022); a report from the AGN SC (August 2022); all these reports are available on the SCOC website at \url{https://lsst.org/content/reports-scs-v2x-simulations}.} Feedback delivered to the SCOC in the form of written reports, through interactions on community.lsst.org,\footnote{Including the thread \url{https://community.lsst.org/t/scoc-v2-0-and-2-1-simulations-review-timeline/6712}.} as well as reported by the SCOC liaisons was considered and incorporated in our decision-making process. 


As discussed above, the comparison of results of the metrics, hereafter referred to as metrics or MAFs, produced by the Survey Strategy team (primarily to assess compliance with the SRD and other system requirements) and by the community constitute a quantitative ---or at least quantifiable--- feedback on LSST simulations. Three kinds of plots are generally produced to compare the performance of simulations for sets of MAFs.

\begin{itemize}

\item {\bf Radar Plots:} helpful to review and compare small numbers of simulations and small numbers of metrics. In a radar plot such as \autoref{fig:radar} each vertex corresponds to a MAF, and each simulation is indicated in a different color. By design, the reference simulation forms a ``perfect'' circle, where the performance for each MAF is =1. For a given simulation, a MAF performance that extends outward (inward) of the reference circle indicates an improvement (decrease in) performance. 

\begin{figure}[h!]
\centering
\includegraphics[height=0.4\textwidth]{figures/SCOC299radar1.png}
\includegraphics[height=0.4\textwidth]{figures/SCOC299radar2.png}
\caption{Radar plots comparing  \texttt{baseline\_v3.0} with earlier baseline simulations (\emph{top}) and the v2.99 simulations released starting on October 24th, 2022 with \texttt{baseline\_v3.0} (\emph{bottom}) . The majority of the metrics considered display improvements compared with earlier baseline simulations. The improvement on the ``XRB early detection'' and ``Microlensing (20\_30 days)'' metrics is $>300\%$ between \texttt{baseline\_v3.0\_10yrs} and \texttt{retro\_baseline\_v2.0\_10yrs} which reproduces baseline \texttt{baseline\_v1.7\_10yrs}, and extend outside of the range of the plot.}\label{fig:radar}
\end{figure}
\FloatBarrier

\item {\bf Heatmaps}: Larger collections of metrics and simulations are generally better visualized with heatmaps (see \autoref{fig:heatmap}), where families of simulations (as columns) and families of MAFs (as rows) can be grouped together by appropriate ordering. 


\begin{figure}[t!]
\centering
\includegraphics[width=0.8\textwidth]{figures/SCOC299heatmap1.png}
\includegraphics[width=0.8\textwidth]{figures/SCOC299heatmap2.png}
\caption{Heat maps comparing the first four v2.99 simulations (released on October 24th, 2022)  with earlier baselines for a selection of system metrics (\emph{top}) and science metrics contributed by the community (\emph{bottom}). Blue blocks indicate improvements compared to a reference simulation (here \texttt{baseline\_v2.0\_10yrs}), and red ones indicate performance loss. The increase or drop of a series of MAFs in adjacent rows may indicate a systemic problem. For example, the SCOC noticed and investigated the decrease in Galactic and Local Volume performance as measured by the number of dwarf galaxies and Young Stellar Objects (YSO) with the help of the SMWLV SC and TVS SC.}
\label{fig:heatmap}
\end{figure}
\FloatBarrier


\item {\bf Line plots}: 
While heatmaps provide a synoptic view, the amount of gain/loss is not obviously quantifiable via the color intensity. Line plots are useful to inspect small numbers of (typically related) MAFs and provide quantifiable significance of a gain/loss (see \autoref{fig:lineplot}). 


\end{itemize}
\begin{figure}[h!]
\centering
\includegraphics[width=0.99\textwidth]{figures/SCOC299lines.png}
\caption{A line plot showing the performance of metrics measuring intra-night cadence for the same simulations used in \autoref{fig:heatmap}: each MAF measures coverage on specific time scales. All values are to be compared with 1, the performance of a reference simulation (here \texttt{baseline\v2.0\_10yrs}). The shaded areas indicate potentially significant performance changes, generally set to a $>5\%$ performance difference. The SCOC notes a generally improved performance for nearly all the v2.99 Time Gap metrics, with performance gains as high as 100\% in some cases.}
\label{fig:lineplot}
\end{figure}

\FloatBarrier
The survey strategy team makes these visualizations available via jupyter notebooks organized by simulation family, MAF family (including notebooks specifically designed for an interest group or SC), and SCOC question (see below). All these notebooks are available on GitHub.\footnote{\url{https://github.com/lsst-pst/survey_strategy}.}



\subsection{Open questions addressed in Phase 2}\label{sec:synopsis}

The SCOC has organized its Phase 2 deliberation around eight topics (posted on community.lsst.org\footnote{\url{https://community.lsst.org/t/scoc-v2-0-and-2-1-simulations-review-timeline/6712} and listed at the end of this section.} in June 2021). Many of these topics came from questions that were addressed to some extent in Phase 1 but needed further study. The open questions and remaining SCOC tasks identified in earlier reports \citepalias{PSTN-051, PSTN-053} were systematically explored in the \texttt{v2.0}, \texttt{v2.1}, and \texttt{v2.2} series of \opsim\ LSST simulations\footnote{The simulations were described in \url{https://github.com/lsst-pst/survey_strategy/blob/main/fbs_2.0/SummaryInfo_v2.1.ipynb} and on community.lsst.org \url{https://community.lsst.org/t/survey-simulations-v2-1-april-2022/6538}.} and include:

%Here we report the questions considered within each topic and the information and deliberations that led to the SCOC recommendations encapsulated in the \texttt{v2.99} and \texttt{baseline\_v3.0} simulations are described in detail in the next section.

\begin{itemize}
\item Establish final survey footprint definitions (\emph{e.g.}, the exact Declination and dust extinction limits for the WFD region, the exact definition of the Galactic bulge region),
\item Decide which sets of filters should be used in sets of visits,
\item Decide the exposure duration and the number of visits in the $u$ band,
\item Optimize further the rolling cadence implementation,
\item Optimize further the DDF cadences.

\end{itemize}

The members of the SCOC further refined the scope of each question and split into eight overlapping subgroups to review the Rubin- and community-contributed metrics relevant to each topic, jointly with community feedback on the simulations. 
The goal of these subgroups was to formulate an initial recommendation to be presented to, reviewed by, and agreed upon by the whole SCOC. Subgroup presentations were distributed over the months of June through August 2022, with the majority of the subcommittee presenting and leading discussions on their recommendations ahead of the August 2022 PCW, where these preliminary recommendations were presented.\footnote{\url{https://project.lsst.org/meetings/rubin2022/agenda/survey-strategy-i}.} Further refinement of the full SCOC deliberations and further review of the recommendations resumed in mid-September 2022.

The eight subcommittees of the SCOC  and the topics they were tasked to address are:
\begin{enumerate}
\item{Early Science} (\autoref{q:Early})

\item{Footprint} (\autoref{q:Footprint})

\item{Filter Distribution} (\autoref{q:Filters})

\item{Nightly Visits pairs and triplets} (\autoref{q:Visits})

\item{Rolling Cadence} (\autoref{q:Rolling})

\item{DDF Strategy} (\autoref{q:DDF})


\item{Microsurveys} (\autoref{q:Micro})

\item{Time allocation for ToO} (\autoref{q:ToO})

\end{enumerate}

In the following section (\autoref{sec:rec}) we describe in detail the deliberation process for each of these eight topics and the associated SCOC Phase 2 recommendation. In \autoref{sec:v3} we describe the \texttt{baseline\_v3.0} simulation that implements our current recommendations. In \autoref{sec:refinements} we highlight the elements of the current recommendation that need further refinement.

%\input{v2.99_sims.tex}
\clearpage


\appendix
% Include all the relevant bib files.
% https://lsst-texmf.lsst.io/lsstdoc.html#bibliographies
% Include all the relevant bib files.
% https://lsst-texmf.lsst.io/lsstdoc.html#bibliographies
\section{Acknowledgements}
The SCOC acknowledges significant efforts and expresses immense gratitute to many dedicated members of the Rubin LSST Science Collaborations and broader scientific community who contributed to the optimization of the Rubin LSST Survey Strategy for many years and continue to actively engage with the SCOC on this topic. 

The SCOC is grateful to the Survey Strategy team who supported the SCOC work and the community contribution to the optimization of the Rubin LSST survey strategy: 
Dr. Peter Yoachim, 
Dr. Eric Neilson,
Dr. Tiago Ribeiro, and Dr. Lynne Jones.

Rubin Observatory Construction Project and Operations acknowledge and thank the SCOC members for their significant and sustained   work toward the design of the LSST observing strategy and in particular express their gratitude to the outgoing members of the committee: Dr. Sarah Brough, Dr. Renée Hložek, Dr. Mansi Kasliwal, Dr. Hiranya Peiris, Dr. Meg Schwamb, Dr. Dan Scolnic.

\subsection{SCOC membership}

During the Phase 2 deliberations, the SCOC composition was as follows:

 Franz Bauer, Universidad Católica, Chile

 Federica Bianco, Rubin Observatory/University of Delaware (ex officio, chair)

 Sarah Brough, University of New South Wales 

 Renee Hlozek, University of Toronto


 Lynne Jones, Rubin Observatory (ex officio)

 Mansi Kasliwal, Caltech 

 Knut Olsen, NSF’s NOIRLab 

 Hiranya Peiris, University College London/Stockholm University

 Meg Schwamb, Queen's University Belfast

 Dan Scolnic, Duke University

 Colin Slater, University of Washington  

 Jay Strader, Michigan State University


\section{References} \label{sec:bib}
\renewcommand{\refname}{} % Suppress default Bibliography section
\bibliography{local,lsst,lsst-dm,refs_ads,refs,books}

% Make sure lsst-texmf/bin/generateAcronyms.py is in your path
\section{Acronyms} \label{sec:acronyms}
\addtocounter{table}{-1}
\begin{longtable}{p{0.145\textwidth}p{0.8\textwidth}}\hline
\textbf{Acronym} & \textbf{Description}  \\\hline

AGN & 
 Active Galactic Nuclei \\
\hline
DDF & Deep Drilling Field \\
\hline
DM & Data Management \\ 
\hline
DESC & Dark Energy Science Collaboration \\
\hline
EDFS & Euclid Deep Field South \\
\hline
LIGO & 
 Laser Interferometer Gravitational-waves Observatory \\
\hline
LMC & 
 Large Magellanic Cloud \\
\hline
LSST & 
 Legacy Survey of Space and Time \\
\hline
MAF & Metric Analysis Framework \\
\hline
NEO & Near Earth Object \\
\hline
NES & North Ecliptic Spur \\
\hline
PCW & Project Community Workshop \\
\hline
PHA & Potentially Hazardous Asteroid  \\ 
\hline
SC & Science Collaboration \\
\hline
SCOC & Survey Cadence Optimization Committee \\
\hline
SCP & South Celestial Pole \\
\hline
SMC & Small Magellanic Cloud \\
\hline
SMWLV & Stars Milky Way Local Volume \\
\hline
SN & Supernova \\
\hline
SRD & Science Requirements Document \\
\hline
SSO & 
 Solar System Object \\
\hline
SSSC & 
 Solar System Science Collaboration \\
\hline
TDE & Tidal Disruption Event \\
\hline
ToO & Target of Opportunity \\
\hline
TVS & Transients and Variable Stars \\
\hline
YSO & Young Stellar Objects \\
\hline
WFD & Wide Fast Deep \\
\hline
\end{longtable}

% If you want glossary uncomment below -- comment out the two lines above
%\printglossaries





\end{document}
