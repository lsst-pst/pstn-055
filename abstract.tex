
This document describes the recommendations formulated by the Survey Cadence Optimization Committee (SCOC) of Vera C. Rubin Observatory for the observing strategy of the Legacy Survey of Space and Time (LSST). These ``Phase 2'' recommendations follow the initial ``Phase 1'' recommendations released in December 2021 as \citeds{PSTN-053} and convey the current recommendations for the initial implementation of the LSST. These recommendations are based on the SCOC analysis of LSST simulations generated according to \citetalias{PSTN-053} and the feedback received from the scientific community on those simulations. These recommendations should be considered as the current guidance for implementing the LSST observing survey at the start of the survey, a strategy to be reviewed regularly over the survey lifetime, and are accompanied by a simulation, \texttt{baseline\_v3.0}, which represents a specific implementation of these recommendations. While these recommendations dramatically narrow the possible LSST observing parameter choices, we emphasize that at this stage they do not yet represent a final or complete set of recommendations for the survey strategy, and several details are still to be refined. Furthermore, the strategy should be considered ``living'' and subject to evolution throughout the survey. The SCOC continues to engage with the community to finalize its recommendations. At the time of the release of this document, the final phases of construction of Rubin Observatory are ongoing with an expected first light in mid-2024. We expect the SCOC recommendations to be revised as needed, as knowledge of the system as built and the outcome of commissioning and Science Verification mature. This document describes the Phase 2 SCOC recommendations ---highlighting elements that can be considered final and elements that remain to be finalized---, the process through which the SCOC has converged on these recommendations, and the working plan for the SCOC starting in January 2023 through the start of the survey to finalize the initial implementation of the Legacy Survey of Space and Time.

